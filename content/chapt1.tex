\bichapter{介绍和使用}{Introduction}\label{chap:intro}
\linespread{1}\selectfont

欢迎使用北京工业大学硕士研究生学位\LaTeX 论文模版(学术型硕士)。制作本模版的思路是因为Office for MacOS的体验极差,加
之本人又对word十分不熟练。网上目前没有现成的模版给我白嫖,本着开源精神和拙劣的\LaTeX 理解,参考部分模版(见致谢)赶工
出该模版的1.0版本。希望借助此模版,使用者能将本就在毕业季不多的精力专注于文字而不是行距、行高、段落等繁琐的格式上。接下来将介绍模版的使用方法。

本模版中大量格式参考《北京工业大学研究生学位论文撰写规范》和北京工业大学博士论文\LaTeX 模版(见致谢)。但仍有小部分
细节格式未确认(见readme todo list)。关于格式定制问题,欢迎issue 至此模版的git仓库。

一些需要你手动设置的内容:

小节之前插入下行语句保证间隔
\begin{center}
  \begin{minipage}{0.5\textwidth}
    \begin{Verbatim}[frame=single]
    \hspace*{\fill}
    \end{Verbatim}
  \end{minipage}
\end{center}

\$M Word的行距计算涉及到不同字体类型,多倍行距单倍行距等内容,甚至可能在不同操作系统下不一致,这和\LaTeX 的基线算法十
分有出入,因此,为了和让人很迷糊的\$M Word 看起来尽可能视觉效果一致,需要手动在每个正文文件前设置如下内容。

\begin{center}
  \begin{minipage}{0.5\textwidth}
    \begin{Verbatim}[frame=single]
\linespread{1.08}\selectfont
    \end{Verbatim}
  \end{minipage}
\end{center}


\hspace*{\fill}

\bisection{如何使用}{howtouse}\label{sec:fst}
\subsection{模版文件总览}
首先要明确的是,\LaTeX 出于安全的考虑,在unix环境下,往上一级目录写入编译时文件(如aux文件)时会被禁止。为了最大兼容
性,当前版本的模版将所有的文件都放置在同一级目录下。(不排除某些tex的特殊编译设置或单独设置的环境变量)

大多数用户使用时,都不必关注所有文件,只需了解如下几个文件:

\begin{center}
  \begin{minipage}{0.95\textwidth}
    \begin{Verbatim}[frame=single]
    主文档 : main.tex         中文摘要 : cabstract.tex
    英文摘要 : eabstract.tex  章节文档 : chapt*.tex
    引用文档 : reference.bib  用户出版物文档 : publication.tex
    致谢文档 : acknowledge.tex
    \end{Verbatim}
  \end{minipage}
\end{center}

以上文档使用时,基本上只需填入文字内容即可,不需要考虑格式。

% 文字\cite{Boutsidis2011}
\subsubsection{主文档}

主文档主要用于记录用户基本信息,组织论文结构。在主文档中,如下内容需要用户自行填写或更改:

\begin{center}
  \begin{minipage}{0.65\textwidth}
    \begin{Verbatim}[frame=single]
      中文图书分类号 clc
      UDC分类号 udc
      学校代码 schoolcode
      密级 secretlevel
      论文中文题目 ctitle
      作者姓名 cauthor
      学科 cdiscipline
      指导教师 csupervisior
      指导教师职称 csupervstitle
      论文提交/答辩日期 cdate
      学号 sutdentid
      论文英文题目 etitle
      研究方向 cmajor
      申请学位 cdegree
      所在单位 ccollege
      授予学位单位 corganization
    \end{Verbatim}
  \end{minipage}
\end{center}

主文档正文部分组织了论文结构,论文正文内容由\verb |chapt1.tex|, \verb|chapt2.tex|等组织。用户自行添加,参考章节文档一节。%(todo)

\subsubsection{中文摘要}

中文摘要内容由\verb|cabstract|环境包裹,两个段落之间以大于1行的空行作为分割。

在中文文档中,有\verb|ckeywords|环境包裹的关键词,以";"分割。

\subsubsection{英文摘要}

 英文摘要与中文摘要一致,由\verb|eabstract|环境包裹。

 类似的,也有\verb|ekeywords|环境包裹的关键词,但以","分割。

 \subsubsection{章节文档}
 章节文档是论文的主要内容,在章节文档中,包含不同的小结,本模版共4级小结,分别由\verb|bisection|, \verb|subsection|, \verb|subsubsection|,  \verb|subsubsubsection| 标记。
 其中,目录会记录至第二级小结,即\verb|subsection|。具体书写格式参考本文档源码(\verb|chapt1.tex|)。

\subsubsection{引用文档}

引用文档格式为正常\verb|bib file|。

\subsubsection{出版物文档}

直接在\verb|publication|环境中添加即可。

\subsubsection{致谢文档}

同引用文档。

\bisection{基本语法}
下面介绍一些基本语法。可自行修改。
\subsection{如何插入图片并引用}

\begin{center}
  \begin{minipage}{0.95\textwidth}
    \begin{Verbatim}[frame=single]
\begin{figure}[htpb]
  \centering
  \includegraphics[width=0.8\textwidth]{bjut_logo_color.pdf}
  \bicaption{中文图题}{English Figure Title}
  \label{fig:logo}
\end{figure}

% 使用\cref{<label>}引用: 如\cref{fig:logo}所示
    \end{Verbatim}
  \end{minipage}
\end{center}

\begin{figure}[htpb]
  \centering
  \includegraphics[width=0.8\textwidth]{bjut_logo_color.pdf}
  \bicaption{中文图题}{English Figure Title}
  \label{fig:logo}
\end{figure}
“如\cref{fig:logo}所示”

\subsection{如何插入表格并引用}

\begin{center}
  \begin{minipage}{0.95\textwidth}
    \small
    \begin{Verbatim}[frame=single]
        \begin{table}[htpb]
          \centering
          \bicaption[]{三线表}{A Table Example}
          \label{tab:1st}
          \begin{tabular}{@{}llr@{}}
            \toprule
            \multicolumn{2}{c}{Item} & \\
            \cmidrule(r){1-2}
            Animal    & Description  & Price(\$) \\
            \midrule
            Gnat      & pergram      & 13.65 \\
            & each         & 0.01 \\
            Gnu       & stuffed      & 92.50 \\
            Emu       & stuffed      & 33.33 \\
            Armadillo & frozen       & 8.99 \\
            \bottomrule
          \end{tabular}
        \end{table}
        % 引用方法与图片一致 如\cref{tab:1st}所示
    \end{Verbatim}
  \end{minipage}
\end{center}

        \begin{table}[htpb]
          \centering
          \bicaption[]{三线表}{A Table Example}
          \label{tab:1st}
          \begin{tabular}{@{}llr@{}}
            \toprule
            \multicolumn{2}{c}{Item} & \\
            \cmidrule(r){1-2}
            Animal    & Description  & Price(\$) \\
            \midrule
            Gnat      & pergram      & 13.65 \\
            & each         & 0.01 \\
            Gnu       & stuffed      & 92.50 \\
            Emu       & stuffed      & 33.33 \\
            Armadillo & frozen       & 8.99 \\
            \bottomrule
          \end{tabular}
        \end{table}

“如\cref{tab:1st}所示”


\subsection{如何插入公式}

注意,插入公式时,公式环境和前后文字之间不应该有空行。

\begin{center}
  \begin{minipage}{0.85\textwidth}
    \begin{Verbatim}[frame=single]
      \begin{equation}\label{eq:eq1}
        e^{\pi i}+1 = 0
      \end{equation}
    \end{Verbatim}
  \end{minipage}
\end{center}

      \begin{equation}\label{eq:eq1}
        e^{\pi i}+1 = 0
      \end{equation}

\subsection{如何插入算法}

\begin{center}
  \begin{minipage}{0.85\textwidth}
    \begin{Verbatim}[frame=single]
\begin{algorithm}[htbp]
  \caption{1+1等于几}
  \label{alg:algorithm}
    \begin{algorithmic}[1]
      \REQUIRE $1$.
      \ENSURE $1$.
      \STATE $i \leftarrow 1$
      \FOR {$i \leq 1$}
        \STATE $i \leftarrow i + 1$;
      \ENDFOR
      \STATE Return $i$.
    \end{algorithmic}
\end{algorithm}
    \end{Verbatim}
  \end{minipage}
\end{center}

\begin{algorithm}[htbp]
\caption{1+1等于几}
\label{alg:algorithm}
\begin{algorithmic}[1]
\REQUIRE $1$.
\ENSURE $1$.
\STATE $i \leftarrow 1$
\FOR {$i \leq 1$}
\STATE $i \leftarrow i + 1$;
\ENDFOR
\STATE Return $i$.
\end{algorithmic}
\end{algorithm}

\begin{assumption}
这是一段假设。
\label{assumpt}
\end{assumption}
如\cref{assumpt}所提到的... ...


\begin{definition}
这是一段定义。
\label{def}
\end{definition}
如\cref{def}所提到的... ...

\begin{proposition}
这是一段命题。
\label{propos}
\end{proposition}
如\cref{propos}所提到的... ...


\begin{lemma}
这是一段引理。
\label{lemma}
\end{lemma}
如\cref{lemma}所提到的... ...

\begin{theorem}
这是一段定理。
\label{theorem}
\end{theorem}
如\cref{theorem}所提到的... ...


\begin{axiom}
这是一段公理。
\label{axiom}
\end{axiom}
如\cref{axiom}所提到的... ...

\begin{lemma}
这是一段引理。
\label{lemma}
\end{lemma}
如\cref{lemma}所提到的... ...

\begin{corollary}
这是一段推论。
\label{corollary}
\end{corollary}
如\cref{corollary}所提到的... ...


\begin{example}
这是一段例子。
\label{example}
\end{example}
如\cref{example}所提到的... ...

\begin{remark}
这是一段注释。
\label{remark}
\end{remark}
如\cref{remark}所提到的... ...

\begin{problem}
这是一段问题。
\label{problem}
\end{problem}
如\cref{corollary}所提到的... ...


\begin{conjecture}
这是一段猜想。
\label{conjecture}
\end{conjecture}
如\cref{conjecture}所提到的... ...
